\subsection{Levantamento} \label{subsection: etapa-LV}

Previamente ao desenvolvimento dos projetos, a contratada deverá proceder ao levantamento com registro fotográfico, gráfico e eletrônico em sistema CAD das áreas de intervenção definidas pela contratante. Conforme a exigência de cada projeto, o levantamento abrangerá a disposição geral da arquitetura (incluindo a metragem quadrada de cada ambiente, dimensões horizontais e verticais, revestimentos e mapa de esquadrias), identificação e localização de todos os pontos de instalação aparentes (incluindo equipamentos prediais), e os elementos estruturais existentes.

A contratada deverá cumprir todas as normas e práticas aplicáveis à um serviço de levantamento de arquitetura, estrutura e instalações. Deverá ser dada especial atenção à Segurança do Trabalho no tocante ao uso de Equipamentos de Proteção Individual (EPI), tais como luvas, máscaras e calçados fechados.

Caso a fiscalização considere inexpressivos quaisquer produtos elaborados ou que eles contenham erros ou ausência de alguma informação, estes serão recusados e a contratada deverá apresentar novos produtos e/ou executar novamente os serviços para nova validação sem ônus para a contratante.

O início do desenvolvimento dos projetos estará condicionado obrigatoriamente à aprovação dos levantamentos realizados pela contratada.

\subsubsection{Generalidades}
	\begin{enumerate}

		\item XXXXXXXXXXXXXXXXXXXXXXXXXX
			\begin{enumerate}
				
				\item XXXXXXXXXXXXXXXXXXXXXXXXXXXXXXXXXXXXXXXXXXXXXXXXXXXXXXXXXX
			
			\end{enumerate}
		
		\item XXXXXXXXXXXXXXXXXXXXXXXXXXXXXXXXXXXXXXX
	\end{enumerate}

\subsubsection{Produtos a serem entregues}
\begin{enumerate}
	
	\item	Conter a confirmação de medidas in loco e de possíveis interferências nas áreas da implantação do novo programa de necessidades;
	
	\item	Registro fotográfico (externo / interno);
	
	\item	Apresentação dos desenhos de as built resultantes dos levantamentos, em escala gráfica apropriada e abordando todos os pavimentos da edificação e seu entorno próximo.
	
	\item Itens mínimos a serem apresentados:
	\begin{enumerate}
		
		\item	Localização e identificação dos quadros elétricos;
		\item	Identificação dos disjuntores e seus respectivos circuitos elétricos;
		\item	Apresentação de diagrama unifilar dos quadros elétricos
		\item	Identificação dos alimentadores gerais dos quadros elétricos;
		\item	Identificação dos alimentadores parciais dos quadros elétricos (quando possível);
		\item	Localização e identificação dos pontos de iluminação (luminárias) e seus respectivos circuitos elétricos;
		\item	Localização de identificação dos pontos de acionamento do sistema de iluminação e seus respectivos circuitos elétricos;
		\item	Localização e identificação dos pontos de tomadas e força e seus respectivos circuitos elétricos;
		\item	Localização e identificação dos pontos de ar-condicionado e seus respectivos circuitos elétricos;
		\item	Localização e identificação da infraestrutura elétrica (eletrocalhas, leitos, perfilados, canaletas no piso e parede);
		\item	Localização e identificação dos eletrodutos (quando possível);
		\item	Localização e identificação de elementos de rede de infraestrutura externa energia elétrica e iluminação pública;
		\item	Localização e identificação dos postes e luminárias externas;
		\item	Localização e identificação de caixa de passagens e caixas de derivações;
		\item	Localização e identificação dos cabos alimentadores; 
		
		
	\end{enumerate}
	
\end{enumerate}