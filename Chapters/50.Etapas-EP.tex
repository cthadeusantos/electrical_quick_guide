\subsection{Estudo preliminar} \label{subsection: etapa-estudo preliminar}

TEXTO A SER ESCRITO EM VERSÃO FUTURA

\subsubsection{Generalidades}

	\begin{enumerate}\label{subsection: EP-generalidades}
		\item Visita técnica ao local de implantação do projeto

		\item Vistoria do entorno e do terreno onde ocorrerá a implantação projeto.
		
		\item Definição dos índices de iluminação a serem adotados.
		
		\item Consulta à legislação pertinente e aos órgãos públicos envolvidos na aprovação do projeto.
	\end{enumerate}

\subsubsection{Levantamento (coleta de dados)}

Uma das fases mais importantes do estudo preliminar é a subetapa de levantamento, onde a equipe de projetistas da contratada estará \textit{"in sito"} onde será desenvolvido o projeto para realizar a coleta de dados e informações necessárias a elaboração do projeto.

\alertwarningbox{
	Uma coleta de dados bem realizada evita revisitas para obtenção de dados não levantados.
}

\begin{enumerate}\label{subsection: levantamento}
	
	\item Levantamento as redes internas de elétrica existentes no local e analisar o impacto causado a elas pela implantação do projeto. 
	
	\item Levantamento as redes externas de elétrica existentes no local e analisar o impacto causado a elas pela implantação do projeto.
	
	\item Levantamento dos serviços públicos existentes.
	
	\item Levantamento de quantidades e potências dos pontos de consumo.
	
	\item Levantamento das cargas.
	
	\item Identificação dos alimentadores gerais dos quadros elétricos.
	
	\item Identificação dos alimentadores parciais dos quadros elétricos (quando possível).

	\item Identificação dos disjuntores e seus respectivos circuitos elétricos.
	
	\item Identificação dos pontos de iluminação (luminárias) e seus respectivos circuitos elétricos.
	
	\item Identificação dos pontos de acionamento do sistema de iluminação e seus respectivos circuitos elétricos.
	
	\item Identificação dos pontos de tomadas e força e seus respectivos circuitos elétricos.
	
	\item Localização e identificação dos pontos de ar-condicionado e seus respectivos circuitos elétricos.
	
	\item Identificação da infraestrutura elétrica (eletrocalhas, leitos, perfilados, canaletas no piso e parede).
	
	\item Identificação dos eletrodutos (quando possível).
	
	\item Identificação de elementos de rede de infraestrutura externa e interna.
	
	\item Identificação dos postes e luminárias externas.
	
	\item Identificação de caixa de passagens e caixas de derivações.
	
	\item Identificação dos cabos alimentadores.

\end{enumerate}

\subsubsection{Produtos a serem entregues}

\paragraph{Relatório preliminar}

\begin{enumerate}
	
	\item Deverão ser apresentados nesta etapa, sob forma de documentos gráficos, os seguintes documentos:
		\begin{enumerate}
			\item Plantas de situação, indicando o terreno e seu entorno imediato onde ocorrerão as intervenções junto à concessionária local.
		
			\item Plantas baixas de localização das edificações, subestação (se aplicável) e demais edículas.
		
			\item Localização e pré-dimensionamento dos equipamentos sugeridos pelo autor do projeto (transformadores, geradores, bombas, etc.).
		\end{enumerate}

	\item Deverão ser apresentados nesta etapa, sob forma de memorial descritivo, os seguintes documentos:
		\begin{enumerate}
			\item Informações levantadas durante a vistoria do local de implantação.
			
			\item Informações obtidas a partir das análises e visitas aos órgãos públicos e concessionárias(se aplicável)
			
			\item Descrição e características da rede pública de fornecimento de energia elétrica; (se aplicável)

			\item Descrição básica do sistema de fornecimento de energia elétrica: entrada, transformação, medição e distribuição - da área de intervenções, distribuição em baixa tensão, iluminação e tomadas, sistema de distribuição de pontos de força, sistema de alarme de segurança, fontes de emergência e pontos de alimentação emergenciais; 

			\item Descrição das informações e características dos aparelhos elétricos vinculados às plantas de layout e com os dados informados pelo usuário; 			

			\item Descrição básica do sistema de aterramento e/ou proteção contra descargas atmosféricas, caso necessário;

			\item Descrição básica das tensões locais de fornecimento de energia elétrica (primária e secundária) encontradas durante o levantamento.
			
			\item Descrição básico das instalações de elétrica atual e descrição dos sistemas propostos pela contratada.
			
			\item Programa de necessidades a partir das informações obtidas.

			\item Informações constando de relatórios fotográficos e pareceres técnicos sobre as instalações e ambientes físicos existentes no local, incluindo análises relativizando as informações recolhidas na etapa do estudo preliminar constando no mínimo as informações adquiridas a partir de \ref{subsection: EP-generalidades}, com o estudo conceitual fornecido pela Fiocruz e com os requisitos técnicos e legais exigidos.
			
			\item Programa básico das instalações elétricas incluindo memória de cálculo preliminar, com justificativa dos sistemas propostos além da determinação dos requisitos e materiais acústicos para atenuação do ruído provocado pelo gerador a ser instalado em ambiente específico. (se aplicável)			
			
			\item Determinação básica dos espaços necessários para as centrais de energia;
			
			\item Determinação básica das áreas destinadas ao encaminhamento horizontal e vertical do sistema elétrico (prumadas);
			
			\item Previsão de consumo de energia elétrica;
			
			\item Pré-localização do sistema de distribuição, prumadas dos leitos/eletrocalhas/eletrodutos e redes em unifilares da alternativa proposta.

			\item Elaboração em formato de tabela estudo comparativo técnico e econômico das alternativas técnicas propostas, aliando preço, facilidade e tempo de execução.
			
			\item Orçamento estimado baseado em projetos anteriores de mesmo porte e especificação desenvolvidos pela contratada
			
			\item Apresentação dos relatórios em arquivo eletrônico (.doc e .pdf) e 01 impressão em formato A4 encadernada e assinada pelo responsável técnico.
			
			\item Apresentação dos documentos gráficos em arquivo eletrônico (.dwg e .pdf) e 01 impressão em formato apropriado assinada pelos profissionais responsáveis.

		\end{enumerate}
	
\end{enumerate}
