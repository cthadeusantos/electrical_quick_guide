\subsection{Estudo preliminar} \label{subsection: estudo preliminar}

TEXTO A SER ESCRITO EM VERSÃO FUTURA

\subsubsection{Generalidades}

\paragraph{Visita técnica ao local de implantação dos projetos}

	\begin{enumerate}
	
		\item Apresentar documento de visita técnica validado por funcionários do setor de manutenção da Fiocruz.
		\item Deverá ser preparado e entregue um documento indicando as áreas visitadas, dias, pessoas contatadas e atas de reuniões com as informações obtidas nessa visita.
		\item Um profissional da CONTRATANTE deverá acompanhar a visita técnica, devendo ser agendada a data e horário de visita. Esse documento deverá ser assinado pelo responsável técnico pelo projeto e pelos funcionários da Fiocruz, lotados nos setores anteriormente citados, que acompanharam a visita do profissional responsável.
	\end{enumerate}

\paragraph{Levantamento das informações básicas sobre o local de implantação do projeto}
	\begin{enumerate}
		\item Relatório com fotos e pareceres técnicos sobre as instalações e ambientes físicos existentes no local, incluindo análises relativizando as informações recolhidas nesta Etapa, com o estudo conceitual fornecido pela Fiocruz e com os requisitos técnicos e legais exigidos.
		
		\item Levantar as redes externas de elétrica existentes no local e analisar o impacto causado a elas pela implantação do projeto. 
		
		\item Programa básico das instalações de elétrica com justificativa e descrição dos sistemas propostos.
		
		\item Elaboração do estudo comparativo técnico e econômico das alternativas técnicas para os sistemas, aliando preço, facilidade e tempo de execução.
		
		\item Complementação da planilha de máquinas e equipamentos para a edificação com a descrição das informações e características dos aparelhos indicando os dados informados pelo usuário. 
		
		\item Apresentação em arquivo eletrônico (.doc e .pdf) e 01 impressão em formato A4 encadernada e assinada pelo responsável técnico.
	\end{enumerate}

\paragraph{Relatório preliminar}
		\begin{enumerate}
			\item Estudo preliminar desenvolvido segundo as normas(??????).
			
			\item Vistoria do entorno e do terreno onde será erguida a edificação.

			\item Levantamento dos serviços públicos existentes.
			
			\item Consulta à legislação pertinente e órgãos públicos envolvidos na aprovação do projeto.
			
			\item Deverão ser apresentados nesta etapa, sob forma de memorial descritivo, os seguintes documentos:
			
			\item Plantas de situação, indicando o terreno e seu entorno imediato onde ocorrerão as intervenções junto à concessionária local.
			
			\item Plantas baixas de localização das edificações, subestação (se aplicável) e demais edículas.
			
			\item Definição dos índices de iluminação a serem adotados.
			
			\item Levantamento de quantidades e potências dos pontos de consumo.
			
			\item Levantamento das cargas.
			
			\item Localização e pré-dimensionamento dos equipamentos sugeridos pelo autor do projeto (transformadores, geradores, bombas, etc.).
			
			\item Definição do sistema de alarme, pontos a serem protegidos e tipos de sensores.
			
			\item Apresentação em arquivo eletrônico (.dwg e .pdf) e 01 impressão em formato apropriado assinada pelos profissionais responsáveis.
		\end{enumerate}

\subsubsection{Produtos a serem entregues}

\begin{enumerate}
	\item Relatório de vistoria do local de implantação atestado por um funcionário da FIOCRUZ.
	\item Relatório das análises das visitas aos órgãos públicos e concessionárias(se aplicável)
	\item Relatório com fotos e pareceres técnicos sobre as instalações e ambientes físicos existentes no local, incluindo análises relativizando as informações recolhidas nesta Etapa, com o estudo preliminar fornecido pela FIOCRUZ e com os requisitos técnicos e legais exigidos. 
	\item Descritivo básico com indicação das alternativas e recomendações de ordem técnica para adequação ao projeto de arquitetura e documentos gráficos para elucidar as proposições técnicas, incluindo, entre outros de ordem legal: Dados da consulta prévia a concessionárias de energia elétrica; (se aplicável)
	\item Programa de necessidades arquitetônicas por ambiente (necessária consulta aos projetos de arquitetura, ar-condicionado, telecomunicações e CFTV).
	\item Programa básico das instalações elétricas incluindo memória de cálculo preliminar, com justificativa dos sistemas propostos além da determinação dos requisitos e materiais acústicos para atenuação do ruído provocado pelo gerador a ser instalado em ambiente específico. (se aplicável)
	\item Localização e características da rede pública de fornecimento de energia elétrica; (se aplicável)
	\item Identificação da tensão local de fornecimento de energia elétrica (primária e secundária); 
	\item Descrição básica do sistema de fornecimento de energia elétrica: entrada, transformação, medição e distribuição - da área de intervenções, distribuição em baixa tensão, iluminação e tomadas, sistema de distribuição de pontos de força, sistema de alarme de segurança, fontes de emergência e pontos de alimentação emergenciais; 
	\item Descrição das informações e características dos aparelhos elétricos vinculados às plantas de layout e com os dados informados pelo usuário; 
	\item Descrição básica do sistema de aterramento e/ou proteção contra descargas atmosféricas, caso necessário;
	\item Determinação básica dos espaços necessários para as centrais de energia;
	\item Determinação básica das áreas destinadas ao encaminhamento horizontal e vertical do sistema elétrico (prumadas);
	\item Previsão de consumo de energia elétrica;
	\item Elaboração do estudo comparativo técnico e econômico das alternativas técnicas para o sistema;
	\item Pré-localização do sistema de distribuição, prumadas dos leitos/eletrocalhas/eletrodutos e redes em unifilares da alternativa proposta.
	\item Planta de locação de iluminação interna na escala 1:50,  indicando: 
	\item localização e especificação dos aparelhos de iluminação, seus comandos; localização dos quadros de distribuição; localização dos pontos de iluminação; e, legenda das convenções usadas.
	\item Planta de locação de iluminação externa na escala 1:50,  indicando: 
	\item localização e especificação dos aparelhos de iluminação, seus comandos; localização dos quadros de distribuição; localização dos pontos de iluminação; e, legenda das convenções usadas.
	\item Planta de locação de iluminação pública na escala 1:50,  indicando: 
	\item localização e especificação dos aparelhos de iluminação, seus comandos; localização dos quadros de distribuição; localização dos pontos de iluminação; e, legenda das convenções usadas.
	\item Planta de locação de tomadas e pontos de força na escala 1:50, indicando: 
	\item localização dos pontos de consumo com as respectivas cargas; localização dos quadros de distribuição e suas respectivas identificações; e, legenda das convenções usadas.
	\item Planta de locação de pontos elétricos de ar-condicionado na escala 1:50, indicando: 
	\item localização dos pontos de consumo com as respectivas cargas, seus comandos; localização dos quadros de distribuição e suas respectivas identificações; e, legenda das convenções usadas.
	\item Definição do sistema e método construtivo das estruturas mais adequadas a todos os projetos aliando preço, facilidade e tempo de execução
	\item Relação quantitativa e qualitativa dos materiais e equipamentos a serem utilizados nos diversos sistemas, contendo: Tipo e qualidade. Características para sua identificação; Unidade de comercialização; Respectivas quantidades. 
\end{enumerate}
