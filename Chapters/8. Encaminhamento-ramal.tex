\subsubsection{Ramal de alimentação dos pavimentos}

\begin{enumerate}
	\item Cada pavimento deverá contar com uma alimentação individual NORMAL, EMERGÊNCIA e ESTABILIZADA (estas duas últimas quando aplicáveis a depender do projeto a ser desenvolvido), sendo previsto um crescimento de 40\% da carga ao longo de 5 anos. Deverá ser considerada a necessidade de um detalhamento no projeto do encaminhamento e interligação do Quadro de Geral de Baixa Tensão ao ponto de entrega do andar;
	
	\item A circulação destes circuitos de alimentação sempre que possível devem ser projetados preferencialmente em leitos de cabos e/ou eletrocalhas nos pavimentos técnicos (se estes existirem no projeto)
	
	\item Em caso de inexistência de pavimento técnicos, o caminhamento de leitos ou eletrocalhas deverá seguir preferencialmente em áreas de circulação comuns;
	
	\item A utilização de salas ou laboratórios para passagens de eletrocalhas/leitos de que conterão ramais de alimentação gerais ou de ramais de alimentação parciais estará proibida e casos excepcionais, a FIOCRUZ deverá obrigatoriamente ser consultada;
	
	\item Deverá ser considerada a necessidade de um detalhamento no projeto do encaminhamento e interligação do Quadro de Geral de Baixa Tensão ao ponto de entrega do andar;

\end{enumerate}