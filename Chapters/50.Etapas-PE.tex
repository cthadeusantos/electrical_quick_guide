\subsection{Projeto executivo} \label{subsection: etapa-projeto executivo}

Consiste na complementação do anteprojeto, contendo todos os detalhamentos das informações e componentes apresentados na etapa anterior.

Para esta etapa, devem ser apresentados os seguintes produtos:

\subsubsection{Produtos a serem entregues}
\alertwarningbox{
	Os produtos devem ser entregues com assinatura digital eletrônica.
	\\
}

\paragraph{Média Tensão / subestação - desenhos}

	\begin{enumerate}
		\item Memória de cálculo para o tratamento acústico para o ambiente do gerador. (se aplicável)
		
		\item Planta de situação na escala 1:250.
		
		\item Planta, corte e elevação da subestação, compreendendo a parte civil e a parte elétrica, na escala 1:50, caso seja necessária sua ampliação. (se aplicável)
		
		\item Planta do ramal de entrada $escala \geq 1:50$ (se aplicável)
		
		\item Planta da subestação $escala \geq 1:25$ (se aplicável)
		
		\item Planta de detalhes construtivos da subestação $escala \geq 1:25$(se aplicável)
		
		\item Planta para aprovação junto a concessionária de energia elétrica do ramal de entrada(se aplicável)
	\end{enumerate}

\paragraph{Ramal de entrada de energia elétrica}
Se este item for aplicável, entregar:
\begin{enumerate}
	\item Conjunto de plantas e especificações técnicas no padrão da concessionária local para obtenção de aprovação legal e concessão da instalação.
	
	\item Abertura do processo de aprovação na concessionária de energia elétrica local com comprovante
	
	\item Entrega do projeto aprovado pela concessionária local
	
	\item Entrega da documentação de aprovação pela concessionária local
\end{enumerate}

\paragraph{Baixa tensão - desenhos}
	\begin{enumerate}

		\item Planta de iluminação interna de todas as edificações divididas por pavimentos / áreas em questão, na escala 1:50, indicando:
			\begin{enumerate}
				\item Traçado, dimensionamento e código de identificação dos condutores e tubulações.

				\item Localização e especificação dos aparelhos de iluminação, seus comandos e indicações dos circuitos pelos quais são alimentados.

				\item Localização dos quadros de distribuição.

				\item Localização dos pontos de iluminação de emergência, iluminação e luz de obstáculos. (se aplicável)

				\item Legenda das convenções usadas.

			\end{enumerate}
		
		\item Planta de iluminação externa de todas as edificações em questão (quando aplicável), na escala 1:50, indicando:
		\begin{enumerate}
			\item Traçado, dimensionamento e código de identificação dos condutores e tubulações.
			
			\item Localização e especificação dos aparelhos de iluminação, seus comandos e indicações dos circuitos pelos quais são alimentados.
			
			\item Localização dos quadros de distribuição.
			
			\item Localização dos pontos de iluminação.
			
			\item Legenda das convenções usadas.
		\end{enumerate}
	
		\item Planta de iluminação pública da área em questão (quando aplicável), na escala 1:100, indicando:
		\begin{enumerate}
			\item Traçado, dimensionamento e código de identificação dos condutores e tubulações.
			
			\item Localização e especificação dos aparelhos de iluminação, seus comandos e indicações dos circuitos pelos quais são alimentados.
			
			\item Localização dos quadros de distribuição.
			
			\item Localização dos pontos de iluminação.
			
			\item Legenda das convenções usadas.
		\end{enumerate}
	
		\item Planta de iluminação cênica (quando aplicável), na escala 1:50, indicando:
			\begin{enumerate}
				\item Traçado, dimensionamento e código de identificação dos condutores e tubulações.
			
				\item Localização e especificação dos aparelhos de iluminação, seus comandos e indicações dos circuitos pelos quais são alimentados.
			
				\item Localização dos quadros de distribuição.
			
				\item Localização dos pontos de iluminação.
			
				\item Legenda das convenções usadas.
			\end{enumerate}
		
		\item Planta de tomadas e pontos de força de todas as edificações divididas por pavimentos, na escala 1:50, indicando:
		\begin{enumerate}

			\item Traçado, distribuição e código de identificação dos circuitos de distribuição, indicando claramente os circuitos de emergência e energia ininterrupta.

			\item Localização dos pontos de consumo com as respectivas cargas, seus comandos e indicações dos circuitos pelos quais são alimentados.

			\item Localização dos quadros de distribuição e suas respectivas identificações.

			\item Identificação dos pontos conectados aos circuitos de emergência. (quando aplicável)
			
			\item Identificação dos pontos conectados aos circuitos de energia ininterrupta. (quando aplicável)
			
			\item Legenda das convenções usadas.
		\end{enumerate}


		\item Planta de pontos de força de ar-condicionados nos pavimentos aplicáveis, na escala 1:50, indicando:
		\begin{enumerate}
			\item Traçado, distribuição e código de identificação dos circuitos de distribuição, indicando claramente os circuitos de emergência.
			
			\item Localização dos pontos de consumo com as respectivas cargas, seus comandos e indicações dos circuitos pelos quais são alimentados.
			
			\item Localização dos quadros de distribuição e suas respectivas identificações.
			
			\item Identificação dos pontos conectados aos circuitos de emergência. (se aplicável)
			
			\item Identificação dos pontos conectados aos circuitos de energia ininterrupta. (se aplicável)
			
			\item Legenda das convenções usadas.
		\end{enumerate}

		\item Planta de encaminhamento dos ramais de alimentação dos quadros elétricos - $escala \geq 1:100$

		\item Planta de encaminhamento da distribuição elétrica do ramal de entrada - $escala \geq 1:50$

		\item Planta do sistema de proteção contra descargas atmosféricas e seus detalhes construtivos - $escala \geq 1:50$.

		\item Plantas dos sistemas de aterramento e seus detalhes construtivos - $escala \geq 1:50$.
		
		\item Esquemas verticais das instalações - prumadas esquemáticas - sem escala

		\item Quadro(s) de carga(s) e detalhes dos quadros de distribuição e dos quadros gerais - $escala \geq 1:25$

		\item Diagramas unifilares gerais .
		
		\item Diagramas trifilares e detalhes dos quadros de distribuição e dos quadros gerais.

		\item Detalhes de interligações, circuitos de comando, suportações, fixações e outros.

		\item Detalhes de execução, montagem e instalações de componentes do sistema, inclusive elementos de suporte, fixação, apoio de tubulações e todos os furos novos necessários nos elementos de estrutura para passagem da instalação, caso necessário.

		\item Memória de cálculo do projeto.

		\item Lista de cabos de força com a identificação de cada circuito, sua origem e destino.

		\item Apresentação em arquivo eletrônico dos desenhos no formato Autocad(.DWG) e Acrobat (pdf) e 01 impressão em formato A4 assinada pelos profissionais responsáveis.
		
		\item Apresentação em arquivo eletrônico dos relatórios, planilhas, dentre outros em formato EXCEL(.xlsx) ou WORD(.docx) e Acrobat (pdf) e 01 impressões em formato A4 assinada pelos profissionais responsáveis.

	\end{enumerate}

\paragraph{Caderno de especificações}
	\begin{enumerate}

		\item Caderno de Especificações completo (Anexo 4) com descrição detalhada dos materiais e características básicas dos principais equipamentos a serem utilizados, incluindo outros elementos que se façam necessários: descrição detalhada e relação qualitativa dos materiais e equipamentos a serem utilizados nos diversos sistemas, contendo: Tipo e qualidade; Características para sua identificação; Unidade de comercialização; processos construtivos e de instalação e de conferências de avaliação; respectivas quantidades.

		\item Caderno de Especificações compatibilizado com todas as disciplinas do projeto do complexo, revisado, atualizado e completo.

		\item Atenção especial deverá ser dada a elaboração da especificação que deverá ser elaborada de acordo com a construção em das etapas apresentadas de acordo com o projeto de arquitetura;

		\item Apresentação em arquivo eletrônico (.doc e .pdf) e 01 impressões em formato A4 assinada pelos profissionais responsáveis.
	\end{enumerate}

\paragraph{Outros itens necessários a aprovação do projeto}
\begin{enumerate}
	\item Memorial de cálculo do projeto, descritivo e explicativo das instalações elétricas ou especiais, indicando fórmulas, dados e métodos utilizados nos dimensionamentos: tensão, corrente, fator de demanda, fator de potência, índice luminotécnico, etc.;

	\item Apresentação dos materiais e equipamentos indicados para aprovação, incluindo, entre outros elementos que se façam necessários: descrição dos materiais e equipamentos a serem utilizados nos diversos sistemas, contendo: Tipo e qualidade; Características para sua identificação; Unidade de comercialização; processos construtivos e de instalação e de conferências de avaliação; respectivas quantidades.

	\item Planilha resumo dos serviços

	\item Planilha de serviços e de materiais com quantitativos e respectivos custos unitários e totais discriminados e orçados (orçamento definitivo).

	\item Planilha da memória da composição dos custos por item de serviço discriminando material, mão-de-obra, encargos e fontes utilizadas.

	\item Planilha de materiais contendo os itens necessários a implementação do projeto, revisado, atualizado e completo.
	
	\item Cronograma físico-financeiro representativo de uma lógica exequível, compatibilização com os projetos e com os quantitativos versus etapas de obra, custos unitários e totais, tempo/períodos de execução mais parcelas de desempenho financeiro relacionadas
	
	\item Apresentação em arquivo eletrônico formato EXCEL(.xlsx) ou WORD(.docx) e Acrobat (pdf) e 01 impressões em formato A4 assinada pelos profissionais responsáveis.

\end{enumerate}