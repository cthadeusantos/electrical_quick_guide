\subsection{Projeto executivo} \label{subsection: etapa-PE}

Consiste na complementação do anteprojeto, contendo todos os detalhes dos componentes das instalações, inclusive elementos de suporte, fixação, apoio de tubulações e furos na estrutura. Deverão ser apresentados os seguintes produtos gráficos:

\paragraph{Projetos}
	\begin{enumerate}

		\item Planta de situação na escala 1:250.

		\item Planta, corte e elevação da subestação, compreendendo a parte civil e a parte elétrica, na escala 1:50, caso seja necessária sua ampliação. (se aplicável)

		\item Planta de iluminação de todos os pavimentos, na escala 1:50, indicando:
			\begin{enumerate}
				\item Traçado, dimensionamento e código de identificação dos condutores e tubulações.

				\item Localização e especificação dos aparelhos de iluminação, seus comandos e indicações dos circuitos pelos quais são alimentados.

				\item Localização dos quadros de distribuição.

				\item Localização dos pontos de iluminação de emergência, iluminação e luz de obstáculos. (se aplicável)

				\item Legenda das convenções usadas.
				Planta de pontos de força de ar-condicionados nos pavimentos aplicáveis, na escala 1:50, indicando:

				\item Traçado, distribuição e código de identificação dos circuitos de distribuição, indicando claramente os circuitos de emergência.

				\item Localização dos pontos de consumo com as respectivas cargas, seus comandos e indicações dos circuitos pelos quais são alimentados.

				\item Localização dos quadros de distribuição e suas respectivas identificações.

				\item Identificação dos pontos conectados aos circuitos de emergência. (se aplicável)

				\item Legenda das convenções usadas.
			\end{enumerate}
		
		\item Planta de tomadas e pontos de força de todos os pavimentos, na escala 1:50, indicando:
		\begin{enumerate}

			\item Traçado, distribuição e código de identificação dos circuitos de distribuição, indicando claramente os circuitos de emergência.

			\item Localização dos pontos de consumo com as respectivas cargas, seus comandos e indicações dos circuitos pelos quais são alimentados.

			\item Localização dos quadros de distribuição e suas respectivas identificações.

			\item Identificação dos pontos conectados aos circuitos de emergência. (se aplicável)
		\end{enumerate}

		\item Legenda das convenções usadas.

		\item Esquemas verticais das instalações.

		\item Quadro(s) de carga.

		\item Diagramas unifilares e detalhes dos quadros de distribuição e dos quadros gerais.

		\item Detalhes de interligações, circuitos de comando, suportações, fixações e outros.

		\item Detalhes de execução, montagem e instalações de componentes do sistema, inclusive todos os furos necessários nos elementos de estrutura para passagem da instalação.

		\item Memória de cálculo do projeto.

		\item Apresentação em arquivo eletrônico (.dwg e .pdf) e 03 impressões em formato A0 assinada pelos profissionais responsáveis.

		\item Lista de cabos de força com a identificação de cada circuito, sua origem e destino.
	\end{enumerate}

\paragraph{Ramal de entrada de energia elétrica}
\begin{enumerate}

\item Conjunto de Plantas e especificações técnicas no padrão da Concessionária local para obtenção de aprovação legal e concessão da instalação.

\item Abertura do processo de aprovação na concessionária de energia elétrica local com comprovante

\item Entrega do projeto aprovado pela Concessionária local

\item Entraga da documentação de aprovação pela Concessionária local
\end{enumerate}


\paragraph{Caderno de especificações}
	\begin{enumerate}

		\item Caderno de Especificações compatibilizado com todas as disciplinas do projeto do complexo, revisado, atualizado e completo.

		\item Atenção especial deverá ser dada a elaboração da especificação que deverá ser elaborada de acordo com a construção em etapas dadas de acordo com o determinado pela arquitetura;

		\item Apresentação em arquivo eletrônico (.doc e .pdf) e 03 impressões em formato A4 assinada pelos profissionais responsáveis.
	\end{enumerate}

\paragraph{Outros itens necessários a aprovação do projeto}
\begin{enumerate}

	\item Caderno de Especificações completo (Anexo 4) com descrição detalhada dos materiais e características básicas dos principais equipamentos a serem utilizados, incluindo outros elementos que se façam necessários: descrição detalhada e relação qualitativa dos materiais e equipamentos a serem utilizados nos diversos sistemas, contendo: Tipo e qualidade; Características para sua identificação; Unidade de comercialização; processos construtivos e de instalação e de conferências de avaliação; respectivas quantidades.

	\item Planilha resumo dos serviços

	\item Planilha de serviços e de materiais com quantitativos e respectivos custos unitários e totais discriminados e orçados (orçamento definitivo).

	\item Planilha da memória da composição dos custos por item de serviço discriminando material, mão-de-obra, encargos e fontes utilizadas.

	\item Planilha de materiais contendo os itens necessários a implementação do projeto, revisado, atualizado e completo.

	\item Cronograma físico representativo de uma lógica exequível das etapas de obra e com todos os projetos compatibilizados

	\item Cronograma físico-financeiro com a compatibilização dos projetos com os quantitativos versus etapas de obra, custos unitários e totais, tempo/períodos de execução mais parcelas de desempenho financeiro relacionadas

	\item Aprovação pela GERENCIADORA / coordenação da FIOCRUZ

	\item Apresentação em arquivo eletrônico formato EXCEL(.xlsx) ou WORD(.docx) e Acrobat (pdf) e 01 impressões em formato A4 assinada pelos profissionais responsáveis.

\end{enumerate}

\subsubsection{Produtos a serem entregues}
	\begin{enumerate}
	
	\item Memorial de cálculo do projeto, descritivo e explicativo das instalações elétricas ou especiais, indicando fórmulas, dados e métodos utilizados nos dimensionamentos: tensão, corrente, fator de demanda, fator de potência, índice luminotécnico, etc.;

	\item Memória de cálculo para o tratamento acústico para o ambiente do gerador. (se aplicável)

	\item Apresentação dos materiais e equipamentos à GERENCIADORA / Coordenação da FIOCRUZ para aprovação, incluindo, entre outros elementos que se façam necessários: descrição dos materiais e equipamentos a serem utilizados nos diversos sistemas, contendo: Tipo e qualidade; Características para sua identificação; Unidade de comercialização; processos construtivos e de instalação e de conferências de avaliação; respectivas quantidades.

	\item Plantas, esquemas e documentos representativos do tratamento acústico para o ambiente do gerador. (se aplicável)

	\item Corte e elevação da alimentação das edificações envolvidas e seus respectivos pavimentos compreendendo desde a derivação do Quadro Geral de Baixa Tensão até o Grupo Motor Gerador de Emergência e os No Breaks - $escala \geq 1:50$ . (se aplicável)

	\item Planta de distribuição dos alimentadores do QGBT, quadros gerais e quadros parciais das edificações envolvidas em seus  respectivos pavimentos na escala 1:50, indicando: 

	\item - traçado, dimensionamento e código de identificação dos condutores e tubulações; localização e especificação dos aparelhos de iluminação, seus comandos e indicações dos circuitos pelos quais são alimentados; localização dos quadros de distribuição; localização dos pontos de iluminação; e, legenda das convenções usadas.

	\item Planta de distribuição de iluminação interna das edificações envolvidas e seus  respectivos pavimentos na escala 1:50, indicando: 

	\item - traçado, dimensionamento e código de identificação dos condutores e tubulações; localização e especificação dos aparelhos de iluminação, seus comandos e indicações dos circuitos pelos quais são alimentados; localização dos quadros de distribuição; localização dos pontos de iluminação; e, legenda das convenções usadas.

	\item Planta de distribuição de iluminação externa na escala 1:50,  indicando: 

	\item traçado, dimensionamento e código de identificação dos condutores e tubulações; localização e especificação dos aparelhos de iluminação, seus comandos e indicações dos circuitos pelos quais são alimentados; localização dos quadros de distribuição; localização dos pontos de iluminação; e, legenda das convenções usadas. (se aplicável)

	\item Planta de distribuição de iluminação pública na escala 1:50,  indicando: 

	\item traçado, dimensionamento e código de identificação dos condutores e tubulações; localização e especificação dos aparelhos de iluminação, seus comandos e indicações dos circuitos pelos quais são alimentados; localização dos quadros de distribuição; localização dos pontos de iluminação; e, legenda das convenções usadas. (se aplicável)

	\item Planta de distribuição de tomadas e pontos de força das edificações envolvidas e seus respectivos pavimentos na escala 1:50, indicando: 

	\item traçado, distribuição e código de identificação dos circuitos de distribuição; localização dos pontos de consumo com as respectivas cargas, seus comandos e indicações dos circuitos pelos quais são alimentados; localização dos quadros de distribuição e suas respectivas identificações; e, legenda das convenções usadas.

	\item Planta de distribuição de tomadas de ar-condicionado das edificações envolvidas e seus respectivos pavimentos na escala 1:50, indicando: 

	\item traçado, distribuição e código de identificação dos circuitos de distribuição; localização dos pontos de consumo com as respectivas cargas, seus comandos e indicações dos circuitos pelos quais são alimentados; localização dos quadros de distribuição e suas respectivas identificações; e, legenda das convenções usadas.

	\item Planta de encaminhamento da distribuição elétrica de iluminação e tomadas interna e externa; escala 1:50

	\item Planta de encaminhamento da distribuição elétrica do ramal de entrada; escala 1:50

	\item Esquemas verticais das instalações - prumadas esquemáticas - sem escala.

	\item Planta do quadro geral de entrada - $escala \geq 1:25$

	\item Diagramas unifilares e trifilares dos quadros elétricos - sem escala 

	\item Planta do ramal de entrada $escala \geq 1:50$ (se aplicável)

	\item Planta da subestação $escala \geq 1:25$ (se aplicável)

	\item Planta de detalhes construtivos da subestação $escala \geq 1:25$(se aplicável)

	\item Planta para aprovação junto a concessionária de energia elétrica do ramal de entrada(se aplicável)

	\item Quadro(s) de carga e detalhes dos quadros de distribuição e dos quadros gerais - $escala \geq 1:25$

	\item Detalhes de execução, montagem e instalações de componentes do sistema, inclusive elementos de suporte, fixação, apoio de tubulações e todos os furos novos necessários nos elementos de estrutura para passagem da instalação, caso necessário.

	\item Finalização do Projeto de instalações elétricas com compatibilização integral com todos os demais projetos, especificações e planilha.

	\item Caderno de Especificações completo (Anexo 4) com descrição detalhada dos materiais e características básicas dos principais equipamentos a serem utilizados, incluindo outros elementos que se façam necessários: descrição detalhada e relação qualitativa dos materiais e equipamentos a serem utilizados nos diversos sistemas, contendo: Tipo e qualidade; Características para sua identificação; Unidade de comercialização; processos construtivos e de instalação e de conferências de avaliação; respectivas quantidades.
	
	\item Planilha resumo dos serviços
	
	\item Planilha de serviços e de materiais com quantitativos e respectivos custos unitários e totais discriminados e orçados (orçamento definitivo).
	
	\item Planilha da memória da composição dos custos por item de serviço discriminando material, mão-de-obra, encargos e fontes utilizadas.
	
	\item Planilha de materiais contendo os itens necessários a implementação do projeto, revisado, atualizado e completo.
	
	\item Cronograma físico representativo de uma lógica exequível das etapas de obra e com todos os projetos compatibilizados
	
	\item Cronograma físico-financeiro com a compatibilização dos projetos com os quantitativos versus etapas de obra, custos unitários e totais, tempo/períodos de execução mais parcelas de desempenho financeiro relacionadas
	
	\item Aprovação pela GERENCIADORA / coordenação da FIOCRUZ
	
	\item Apresentação em arquivo eletrônico formato EXCEL(.xlsx) ou WORD(.docx) e Acrobat (pdf) e 01 impressões em formato A4 assinada pelos profissionais responsáveis.
	\end{enumerate}