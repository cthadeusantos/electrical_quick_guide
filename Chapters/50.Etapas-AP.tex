\subsection{Anteprojeto} \label{subsection: anteprojeto}

Consiste na solução definitiva do estudo preliminar, depois de absorvidas as alterações e complementações feitas durante a análise do projeto elaborado, incluindo a coordenação do início dos projetos complementares e compatibilizando-os com o projeto arquitetônico.

Corresponde à identificação das interfaces entre as diversas disciplinas mais as determinações de soluções e definição técnicas para a elétrica, ou seja, corresponde ao aprofundamento das soluções técnicas conjugadas e ao desdobramento do que foi aprovado na etapa anterior.

O anteprojeto de elétrica deve apresentar em suas representações bidimensionais (plantas e cortes) ou tridimensionais, a compatibilização com todas as demais disciplinas do projeto do empreendimento.
Consiste no dimensionamento do sistema adotado e na localização precisa de seus componentes.

\subsubsection{Generalidades}
Produtos a serem apresentados:
\paragraph{Projeto}
\begin{enumerate}
		\item Planta(s) de iluminação de todos os pavimentos, na escala 1:50, indicando:
		\begin{enumerate}
			\item Traçado e dimensionamento dos circuitos de distribuição.
			\item Localização dos quadros de distribuição.
			\item Localização dos aparelhos de iluminação com indicação das suas características.
			\item Dimensionamento e layout da subestação e sala dos geradores. (se aplicável)
			\item Localização do pára-raios.
			\item Localização e desenvolvimento dos sistemas de aterramento.
			\item Planta de alarme de todos os pavimentos indicando o traçado do sistema, dimensionamento dos eletrodutos e cabos, localização do painel de sinalização e controle.
			\item Planta(s) de pontos de força de ar-condicionado de todos os pavimentos aplicáveis, na escala 1:50, indicando:
			\item Traçado e dimensionamento dos circuitos de distribuição.
			\item Localização dos quadros de distribuição.
			\item Localização dos pontos de consumo com as respectivas cargas.
			\item Planta(s) de tomadas e pontos de força de todos os pavimentos, na escala 1:50, indicando:
				\begin{enumerate}
					\item Traçado e dimensionamento dos circuitos de distribuição.
					\item Localização dos quadros de distribuição.
					\item Localização dos pontos de consumo com as respectivas cargas.
				\end{enumerate}
		\end{enumerate}
	
		\item Apresentação em arquivo eletrônico (.dwg e .pdf) e 01 impressão em formato A0 assinada pelos profissionais responsáveis.	
\end{enumerate}

\paragraph{Caderno de especificações técnicas}
\begin{enumerate}
	\item Apresentação preliminar do Caderno de Especificações com as características básicas dos principais equipamentos a serem utilizados.
	\item Apresentação em arquivo eletrônico (.doc e .pdf) e 01 impressão em formato A4 assinada pelos profissionais responsáveis
\end{enumerate}

\paragraph{Orçamento preliminar}
\begin{enumerate}
	\item Apresentação do orçamento preliminar.
	\item Apresentação em arquivo eletrônico (.doc e .pdf) e 01 impressão em formato A4 assinada pelos profissionais responsáveis
\end{enumerate}

\subsubsection{Produtos a serem entregues}
	\begin{enumerate}
		
			\item Relação quantitativa e qualitativa dos materiais e equipamentos a serem utilizados nos diversos sistemas, contendo: Tipo e qualidade. Características para sua identificação; Unidade de comercialização; Respectivas quantidades. 


			\item Complementação da planilha de máquinas e equipamentos para a edificação com a descrição das informações e características dos aparelhos indicando os dados informados pelo usuário. 

	\item Planta de locação de iluminação interna na escala 1:50,  indicando: 

\item localização e especificação dos aparelhos de iluminação, seus comandos; localização dos quadros de distribuição; localização dos pontos de iluminação; e, legenda das convenções usadas.

\item Planta de locação de iluminação externa na escala 1:50,  indicando: 

\item localização e especificação dos aparelhos de iluminação, seus comandos; localização dos quadros de distribuição; localização dos pontos de iluminação; e, legenda das convenções usadas.

\item Planta de locação de iluminação pública na escala 1:50,  indicando: 

\item localização e especificação dos aparelhos de iluminação, seus comandos; localização dos quadros de distribuição; localização dos pontos de iluminação; e, legenda das convenções usadas.

\item Planta de locação de tomadas e pontos de força na escala 1:50, indicando: 

\item localização dos pontos de consumo com as respectivas cargas; localização dos quadros de distribuição e suas respectivas identificações; e, legenda das convenções usadas.

\item Planta de locação de pontos elétricos de ar-condicionado na escala 1:50, indicando: 

\item localização dos pontos de consumo com as respectivas cargas, seus comandos; localização dos quadros de distribuição e suas respectivas identificações; e, legenda das convenções usadas.
		
		
		\item Memorial de cálculo do projeto, descritivo e explicativo das instalações elétricas ou especiais, indicando fórmulas, dados e métodos utilizados nos dimensionamentos: tensão, corrente, fator de demanda, fator de potência, índice luminotécnico, etc.;
		\item Memória de cálculo para o tratamento acústico para o ambiente do gerador.
		\item Apresentação dos materiais e equipamentos à GERENCIADORA / Coordenação FIOCRUZ para aprovação, incluindo, entre outros elementos que se façam necessários: descrição dos materiais e equipamentos a serem utilizados nos diversos sistemas, contendo: Tipo e qualidade; Características para sua identificação; Unidade de comercialização; processos construtivos e de instalação e de conferências de avaliação; respectivas quantidades.
		\item Plantas, esquemas e documentos representativos do tratamento acústico para o ambiente do gerador.
		\item Planta de distribuição de iluminação interna na escala 1:50,  indicando: 
		\item traçado, dimensionamento e código de identificação dos condutores e tubulações; localização e especificação dos aparelhos de iluminação, seus comandos e indicações dos circuitos pelos quais são alimentados; localização dos quadros de distribuição; localização dos pontos de iluminação; e, legenda das convenções usadas.
		\item Planta de distribuição de iluminação externa na escala 1:50,  indicando: 
		\item traçado, dimensionamento e código de identificação dos condutores e tubulações; localização e especificação dos aparelhos de iluminação, seus comandos e indicações dos circuitos pelos quais são alimentados; localização dos quadros de distribuição; localização dos pontos de iluminação; e, legenda das convenções usadas.
		\item Planta de distribuição de iluminação pública na escala 1:50,  indicando: 
		\item traçado, dimensionamento e código de identificação dos condutores e tubulações; localização e especificação dos aparelhos de iluminação, seus comandos e indicações dos circuitos pelos quais são alimentados; localização dos quadros de distribuição; localização dos pontos de iluminação; e, legenda das convenções usadas. (se aplicável)
		\item Planta de distribuição de tomadas e pontos de força na escala 1:50, indicando: 
		\item traçado, distribuição e código de identificação dos circuitos de distribuição; localização dos pontos de consumo com as respectivas cargas, seus comandos e indicações dos circuitos pelos quais são alimentados; localização dos quadros de distribuição e suas respectivas identificações; e, legenda das convenções usadas.
		\item Planta de distribuição de pontos elétricos de ar-condicionado na escala 1:50, indicando: 
		\item traçado, distribuição e código de identificação dos circuitos de distribuição; localização dos pontos de consumo com as respectivas cargas, seus comandos e indicações dos circuitos pelos quais são alimentados; localização dos quadros de distribuição e suas respectivas identificações; e, legenda das convenções usadas.
		\item Planta de encaminhamento da distribuição elétrica de iluminação e tomadas interna e externa; escala 1:50
		\item Planta de encaminhamento da distribuição elétrica do ramal de entrada; escala 1:50
		\item Planta do quadro geral de entrada - $escala \geq 1:25$
		\item Planta do ramal de entrada $escala \geq 1:50$
		\item Planta da subestação $escala \geq 1:25 $ (se aplicável)
		\item Planta para aprovação junto a concessionária de energia elétrica do ramal de entrada(se aplicável)
		\item Quadro(s) de carga e detalhes dos quadros de distribuição e dos quadros gerais - $escala \geq 1:25$
		\item Apresentação preliminar do Caderno de Especificações com descrição e relação qualitativa dos materiais e equipamentos a serem utilizados nos diversos sistemas, contendo: Tipo e qualidade; Características para sua identificação; Unidade de comercialização e de conferências de avaliação;
	\end{enumerate}
