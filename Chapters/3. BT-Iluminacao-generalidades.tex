\subsubsection{Generalidades} \label{lighting - generalidades}

\begin{enumerate}
	
	\item O projeto de iluminação deverá abranger, onde cabível, os seguintes sistemas:
	\begin{enumerate}
		\item Iluminação geral de interiores
		\item Iluminação externa
		\item Iluminação específica
		\item Iluminação de emergência
		\item Iluminação de sinalização e luz de obstáculos
		\item Iluminação cênica (quando esta for exigida)
	\end{enumerate}
	
	\item Não será aceita a utilização de eletrodutos de bitola menor que 3/4” de diâmetro para iluminação.
	
	\item Poderá ser considerada a instalação como previsão de reserva, eletrodutos com bitolas superiores às necessárias para as bitolas iniciais dos condutores ou eletrodutos vazios.
	
	\item O projeto deverá priorizar, sempre que possível, a utilização de luminárias energeticamente eficientes.
	
	\item O projeto sempre deverá priorizar a utilização de equipamentos e materiais facilmente encontrados no mercado.
	
	\item Deverá ser priorizada a utilização de luminárias na seguinte ordem:\label{light: tipo1}
	\begin{enumerate}
		\item Utilização de luminárias LED com lâmpadas TUBOLED modelo T5, dado que as lâmpadas são facilmente substituídas.

		\item Luminárias com lâmpadas LED (a lâmpada ou o conjunto de lâmpadas pode ser facilmente substituído);
		
		\item Luminárias LED (o conjunto de LEDs não pode ser substituído, entretanto o \textit{driver} pode ser substituído;
	\end{enumerate}

	\item Deverá ser evitada, entretanto poderá ser utilizada em casos excepcionais:\label{light: tipo2}
	\begin{enumerate} 
		
		\item Luminárias LED cujo conjunto de lâmpadas e driver não podem ser substituídos (a manutenção deverá trocar todo o conjunto em caso de defeito)		
	\end{enumerate}

	\item Casos que não se enquadrem nos itens \ref{light: tipo1} e \ref{light: tipo2}, a contratante deverá ser contatada para definir junto a contratada quais luminárias e lâmpadas serão utilizadas.

	\item O projeto de iluminação atenderá aos níveis de iluminamento necessários em cada ambiente de acordo com a NBR-8.995 e determinará o tipo de iluminação, número de lâmpadas por luminárias, número e tipo de luminária, detalhes de montagem, localização das luminárias, caixas de passagem e interruptores, caminhamento dos condutores e tipo para sua instalação. 
	
	\item Para o projeto de iluminação poderão ser adotados os valores mínimos dos níveis de iluminamento recomendados pelas NBR 8.995. O tipo de fonte luminosa e da luminária e a sua distribuição no local deverão ser harmonizados com os projetos de arquitetura e aprovados pela coordenação do desenvolvimento do projeto.
	
	\item \label{lighting: bitola minima} Para circuitos de iluminação, será adotado a bitola mínima de 2,5mm\textsuperscript{2} observando-se, entretanto, a diferenciação de cores nas respectivas fiações.
	
	\item As derivações para a alimentação das luminárias serão executadas utilizando-se cabos de cobre “PP” de 3 vias \#1,5mm\textsuperscript{2} devendo ser indicadas em nota e preferencialmente em detalhe de projeto;
	
	\item Em instalações aparentes deverá ser utilizada as linhas disponíveis da Wetzel como referência técnica para interruptores e tampas montados em conduletes de alumínio
	
	\item A contratante utiliza alguns modelos de luminária (interna, externa e pública) como padrão de referência técnica. O projetista deverá consultar o apêndice XXXXXX para obter estes modelos ou apresentar à Engenharia da contratante para aprovação os modelos com características similares que pretende utilizar no projeto.
	
	\item Ao dimensionar os eletrodutos de tomadas, os mesmos deverão ser dimensionados com bitola mínima de 3/4"%$\dfrac{3}{4}$"
\end{enumerate}